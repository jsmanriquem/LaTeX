% Inserte el nombre del espacio académico:

\nombre{Física 1 y laboratorio}

% Inserte el código del espacio:

\codigo{25102}

% Inserte el número de créditos:

\creditos{3}

% Seleccione el tipo de curso, rellene el primer corchete para el tipo de
% curso teórico, el segundo para el tipo de curso práctico, y el último 
% para el tipo de curso teórico-práctico.

\curso{}{}{x}

% Seleccione el tipo de espacio académico, rellene el primer corchete para
% sí, o rellene el segundo corchete para no.

\obligatoriobasico{x}{} % Obligatorio básico
\obligatoriocomplementario{}{x} % Obligatorio complementario
\electivointrinseco{}{x} % Electivo intrínseco
\electivoextrinseco{}{x} % Electivo extrínseco

% Indique el número de horas pertinente para cada trabajo:

\trabajodirecto{4} % Trabajo directo
\trabajomediano{2} % Trabajo mediano
\trabajoautonomo{3} % Trabajo autónomo

% A continuación, coloque la justificación, si usted precisa de un salto
% de línea pero no con un espacio entre medias coloque \\, si necesita el
% espacio, coloque \\ \\.

\justificacion{Física I y Laboratorio es el primero de una secuencia de tres cursos para estudiantes con formación en ciencias y/o ingeniería. En este curso, en primera instancia,  se desarrollan los elementos fundamentales de la descripción del movimiento de una partícula;  en una y dos dimensiones espaciales, y en el tiempo;   con trayectorias: lineal , cuadrática y circular. Luego se presentan los elementos fundamentales de la mecánica de Newton: Definiciones (Punto material, Sistema de Referencia (Observador, Aparatos de Medida y Sistema de Coordenadas), desplazamiento espacial y temporal, velocidad y aceleración), Leyes (Ley de Inercia, Ley de Fuerza y Ley de Acción Reacción) y Teoremas (Teorema de la Conservación de la Energía, Conservación del Momento Lineal y Conservación del Momento Angular).  Finalmente se realiza un acercamiento sucinto a la Mecánica de muchas partículas, específicamente a la mecánica del sólido rígido.
\\ \\
En este espacio académico se busca obtener resultados de aprendizaje asociados a las matemáticas y comprensión   del quehacer científico, además de ser una oportunidad de obtener una comprensión de los principios físicos fundamentales en la mecánica clásica. Las prácticas de laboratorio propuestas   permiten materializar y verificar, los principios básicos mecánica clásica, a través del manejo de equipos de medida y manipulación de los datos obtenidos (recolección, organización, representación gráfica , ajuste y análisis)  por medio de la  reproducción de experimentos fundamentales de la mecánica clásica enmarcados en el contexto del ahora.
}

% A continuación, coloque la programación del contenido, para ello ubique
% cada item en frente del comando \item que puede ver de ejemplo, si
% requiere más, solamente agregue el comando junto a su contenido.
\contenido{
    \item Introducción a las Ciencias: Física.
    \item Unidades, Cantidades Físicas y Vectores.
    \item Descripción del movimiento de la partícula en línea recta con aceleración y velocidad constantes.
    \item Descripción del movimiento de la partícula en dos dimensiones.
    \item Leyes del Movimiento de Newton de una partícula y aplicaciones. Ley de Gravitación Universal.
    \item Definición de trabajo, energía cinética, energía potencial. Teorema de la Conservación de la Energía.
    \item Definición de momento lineal, impulso. Teorema de la Conservación del Momento Lineal. Choques elásticos e inelásticos. 
    \item Definición de momento angular, impulso. Teorema de la Conservación del Momento Angular.
    \item Mecánica de un sistema de partículas. Definición de cuerpo rígido. Dinámica y cinemática de rotación de un cuerpo rígido.
}

% A continuación, coloque las estrategias, si usted precisa de un salto
% de línea pero no con un espacio entre medias coloque \\, si necesita el
% espacio, coloque \\ \\, además, si necesita un entorno de enumeración,
% invoque el entorno 'enumerate' y en los items coloque lo que precise.

\estrategias{\textbf{Metodología pedagógica y didáctica:} \\
Métodos Instructivos: Los métodos incluirán conferencias y clases magistrales,  que analizan términos clave, conceptos y fórmulas del tema abordado. Durante la conferencia se espera introducir a las leyes y teorías propias del tema, para luego como trabajo extra-clase el estudiante refuerce sus entendimientos con los conceptos claves del tema, esto junto con problemas asignados en cada sesión permitirán un desarrollo progresivo en cada uno de los contenidos del curso. Para una evolución gradual es indispensable un trabajo extra-clase (horas de estudio fuera del aula cada sesión). Los problemas asignados previamente, los abordados en clase, junto con los laboratorios serán las bases del objetivo final del curso. Este proceso está diseñado para ayudar al estudiante a comprender a fondo los conceptos y aplicaciones del material cubierto. \\ \\
\textbf{Métodos de evaluación:}
\begin{enumerate}
    \item Exámenes Parciales.
    \item Tareas asignadas.
    \item Informes de laboratorio.
\end{enumerate}
}
